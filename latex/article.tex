\documentclass[11pt,a4paper]{article}
\linespread{1.25}
\usepackage[utf8]{inputenc} % Norwegian letters
\usepackage{natbib}
\usepackage{amsmath}
\usepackage{float}  % Used for minipage and stuff.
\usepackage{wrapfig} % Wrap text around figure wrapfig [tab]
\usepackage{graphicx}
\usepackage{enumerate} % Use e.g. \begin{enumerate}[a)]
\usepackage[font={small, it}]{caption} % captions on figures and tables
\usepackage{subcaption}
\usepackage[toc,page]{appendix} % Page make Appendices title, toc fix table of content 
%\usepackage{todonotes} % Notes. Use \todo{"text"}. Comment out \listoftodos
\usepackage{xargs}                      % Use more than one optional parameter in a new commands
\usepackage[pdftex,dvipsnames]{xcolor}  % Coloured text etc.
% 
\usepackage[colorinlistoftodos,prependcaption,textsize=footnotesize]{todonotes}
\newcommandx{\unsure}[2][1=]{\todo[linecolor=red,backgroundcolor=red!25,bordercolor=red,#1]{#2}}
\newcommandx{\change}[2][1=]{\todo[linecolor=blue,backgroundcolor=blue!25,bordercolor=blue,#1]{#2}}
\newcommandx{\info}[2][1=]{\todo[linecolor=OliveGreen,backgroundcolor=OliveGreen!25,bordercolor=OliveGreen,#1]{#2}}
\newcommandx{\improvement}[2][1=]{\todo[linecolor=Plum,backgroundcolor=Plum!25,bordercolor=Plum,#1]{#2}}
\newcommandx{\thiswillnotshow}[2][1=]{\todo[disable,#1]{#2}}
\usepackage{microtype} % Improves spacing. Include AFTER fonts
\usepackage{hyperref} % Use \autoref{} and \nameref{}
\hypersetup{backref,
  colorlinks=true,
  breaklinks=true,
  %hidelinks, %uncomment to make links black
  citecolor=blue,
  linkcolor=blue,
  urlcolor=blue
}
\usepackage[all]{hypcap} % Makes hyperref jup to top of pictures and tables
%
%-------------------------------------------------------------------------------
% Page layout
%\usepackage{showframe} % Uncomment if you want the margin frames
\usepackage{fullpage}
\topmargin=-0.25in
%\evensidemargin=-0.3in
%\oddsidemargin=-0.3in
%\textwidth=6.9in
%\textheight=9.5in
\headsep=0.25in
\footskip=0.50in

%-------------------------------------------------------------------------------
% Header and footer
\usepackage{lastpage} % To be able to add last page in footer.
\usepackage{fancyhdr} % Custom headers and footers
\pagestyle{fancy} % Use "fancyplain" for header in all pages
%\renewcommand{\chaptermark}[1]{ \markboth{#1}{} } % Usefull for book?
\renewcommand{\sectionmark}[1]{ \markright{\thesection\ #1}{} } % Remove formating and nr.
%\fancyhead[LE, RO]{\footnotesize\leftmark}
%\fancyhead[RO, LE]{\footnotesize\rightmark}
\lhead[]{\AuthorName}
\rhead[]{\rightmark}
\fancyfoot[L]{} % Empty left footer
\fancyfoot[C]{} % Empty center footer
\fancyfoot[R]{Page\ \thepage\ of\ \protect\pageref*{LastPage}} % Page numbering for right footer
\renewcommand{\headrulewidth}{1pt} % header underlines
\renewcommand{\footrulewidth}{1pt} % footer underlines
\setlength{\headheight}{13.6pt} % Customize the height of the header

%-------------------------------------------------------------------------------
% Suppose to make it easier for LaTeX to place figures and tables where I want.
\setlength{\abovecaptionskip}{0pt plus 1pt minus 2pt} % Makes caption come closer to figure.
%\setcounter{totalnumber}{5}
%\renewcommand{\textfraction}{0.05}
%\renewcommand{\topfraction}{0.95}
%\renewcommand{\bottomfraction}{0.95}
%\renewcommand{\floatpagefraction}{0.35}
%
% Math short cuts for expectation, variance and covariance
\newcommand{\E}{\mathrm{E}}
\newcommand{\Var}{\mathrm{Var}}
\newcommand{\Cov}{\mathrm{Cov}}
\newcommand{\mk}[1]{\colorbox{yellow}{#1}}
% Commands for argmin and argmax
\DeclareMathOperator*{\argmin}{arg\,min}
\DeclareMathOperator*{\argmax}{arg\,max}
%%%%%%%%%%%%%%%%%%%%%%%%%%%%%%%%%%%%%%%%%%%%%%%%%%%%%%%%%%%%%%%%%%%%%%%%%%%%%%%%
%-----------------------------------------------------------------------------
%	TITLE SECTION
%-----------------------------------------------------------------------------
\newcommand{\AuthorName}{Håvard Kvamme} % Your name

\newcommand{\horrule}[1]{\rule{\linewidth}{#1}} % Create horizontal rule command with 1 argument of height

\title{\
\normalfont \normalsize 
\textsc{STK9101SP} \\ [25pt] % Your university, school and/or department name(s)
\horrule{0.5pt} \\[0.4cm] % Thin top horizontal rule
\huge Project on Survival Analysis \\ % The assignment title
\horrule{2pt} \\[0.5cm] % Thick bottom horizontal rule
}

\author{\AuthorName} % Your name

\date{\normalsize\today} % Today's date or a custom date
\begin{document}
\maketitle
%%%%%%%%%%%%%%%%%%%%%%%%%%%%%%%%%%%%%%%%%%%%%%%%%%%%%%%%%%%%%%%%%%%%%%%%%%%%%%%%
%\listoftodos{}
%
\section{Introduction}
\todo[inline]{Introduce problem and covariates}
This project is the first part of the final exam in STK4080/9080.

Primary biliary cirrhosis, or PBC, is a rare liver decease. We do no currently know what cause this decease.  
The American hospital Mayo followed 307 PBC patients in a double-blind study from January 1974 to May 1984, where the drug D-penicillamine was compared to placebo.
Numerous covariates were recorded for the patients at the start of the studey, however, we will only work with a small subset of those.
We have chosen to focus on the demographic variables gender and age, and the two biochemical covariates albumin (the dominating protein in blood plasma) and bilirubin (a yellow compompound that, in large quantities, is related with liver malfunction).

\begin{itemize}
    \item \verb+days+: Days from randomization to death or censoring. 
    \item \verb++: 
    \item \verb++: 
    \item \verb++: 
\end{itemize}

The dataset has tied events due to the rounding of events to nearest date. Out of the 307 observations, 296 occurs alone in a single date. However, this does not affect the estimation of the Kaplan-Meier. 
For Cox regression however, ties can be a problem. There are different ways to handle ties here, but as there are very few ties compared to the size of the dataset, the different methods should yield more or less equivalent results. We therefore only use the Efron approximation in all our experiments.


\section{a) Simple univariate analysis}
\unsure[inline]{Should I try to interpret the K-M plots through quantiles and stuff?? I’m pretty sure I’ve seen that somewhere. Show median survival times for different groups?}

To start with, we analyze the relationship between the death rates and the covariates, individually.

Figure~\ref{fig:km_beh_ald},~\ref{fig:km_kjonn_bil}, and~\ref{fig:km_alb} shows the survival curves (Kaplan-Meier plots) for the individual covariates. 
Under each plot we have added a histogram for the covariates. This is helpful for choosing the binning of numerical covariates.

To the left in Figure~\ref{fig:km_beh_ald} we see that there seems to be very little difference between the treatment and placebo patients. There nothing here that suggest that treatment has a positive effect. Table~\ref{tab:log_rank_indiv} shows the log-rank test for the individual covariates, and we see here that treatment is not considered significant.

To the right in Figure~\ref{fig:km_beh_ald} we see the survival curves for different age groups. There is quite a large difference between the oldest patients and the younger patients, suggesting that age play an important part in the death of the PBC patients. We also the that the age groups are considered significant by the log-rank test in Table~\ref{tab:log_rank_indiv}.
However, I do not know how deaths are recorded in the dataset. If deaths cause by something not related to the PBC are recorded as deaths, it is possible that we only see the effect age has on humans in general.  To truly say something about age, we should have a group of individuals without PBC and compare the death rates between the groups.
%
\begin{figure}[h!tbp]
    \centering
    \begin{subfigure}[b]{0.48\textwidth}
        \includegraphics[width=\textwidth]{./figures/km_beh.pdf}
        %\caption{Alb.}
        %\label{fig:cartAreas1}
    \end{subfigure}%
    \quad
    \begin{subfigure}[b]{0.48\textwidth}
        \includegraphics[width=\textwidth]{./figures/km_ald.pdf}
        %\caption{not alb.}
        %\label{fig:cartTree1}
    \end{subfigure}
    %(or a blank line to force the subfigure onto a new line)
    \vspace{1\baselineskip}
    \caption{Survival curves for treatment and age. We have included histograms for the two covariates under the survival curves.}
    \label{fig:km_beh_ald}
\end{figure}

In Figure~\ref{fig:km_kjonn_bil}, we see the survival curves for gender and bilirubin. The figure might suggest that males has a higher hazard than females, but we see from the bar chart that there are quite few males in the dataset, so the difference might just be a results of small sample size. The log-rank test supports this hypothesis, as the $p$-value is quite small, but not very small.

Bilirubin, however, seem to have a possibly large effect on the survival. Comparing the death ratios between the groups with the highest and lowest concentration of bilirubin, the difference is quite striking. Of course bilirubin is considered significant by the log-rank test.
%
\begin{figure}[h!tbp]
    \centering
    \begin{subfigure}[b]{0.48\textwidth}
        \includegraphics[width=\textwidth]{./figures/km_kjonn.pdf}
        %\caption{Alb.}
        %\label{fig:cartAreas1}
    \end{subfigure}%
    \quad
    \begin{subfigure}[b]{0.48\textwidth}
        \includegraphics[width=\textwidth]{./figures/km_bil.pdf}
        %\caption{not alb.}
        %\label{fig:cartTree1}
    \end{subfigure}
    %(or a blank line to force the subfigure onto a new line)
    \vspace{1\baselineskip}
    \caption{Survival curves for gender and amount of bilirubin in the blood of the test subjects. We have included histograms for the two covariates under the survival curves.}
    \label{fig:km_kjonn_bil}
\end{figure}

Finally, we investigated the relationship between the concentration of albumin and the survival function in Figure~\ref{fig:km_alb}. Immediately, it is apparent that individuals with a lower concentration, has higher death rates. This is surprising, as a high concentration of albumin is usually connected to lowered functionality in the liver (see assignment text).
The log-rank test also considers albumin as very significant.

\begin{figure}[h!tb]
    \begin{center}
        \includegraphics[scale=0.8]{./figures/km_alb.pdf}
    \end{center}
    \vspace{-0.8cm}
    \caption{Survival curve for amount of albumin in the blood of the test subjects. The histogram displays the distribution of albumin among the subjects.}
    \label{fig:km_alb}
\end{figure}

\begin{table}[h!tbp]
    \centering
    \caption{Log-rank tests on the individual covariates.}
    \label{tab:log_rank_indiv}
    \begin{tabular}{rrrr}
        \hline
        Covariate & deg. & Chisq. & $p$ \\ 
        \hline
        Treatment &  1 &  $0   $ & $0.853     $ \\
        Age       &  3 &  $21  $ & $1.07e-4   $ \\ 
        Gender    &  1 &  $3.5 $ & $0.0613    $ \\ 
        Bilirubin &  4 &  $182 $ & $0         $ \\ 
        Albumin   &  3 &  $68.7$ & $7.99e-15$ \\ 
        \hline
    \end{tabular}
\end{table}

\section{b) Univariate regression}
\info[inline]{For categorical covariates, cox scores and log-rank are equivalent, if not ties. What with ties??}
\unsure[inline]{How does cox handle tied data?}
\info[inline]{Cox can fail if $x$ is wrong and we should use log or sqrt instead. Or if the hazards are not proportional.}

Second, we analyzed the individual covariates by univariate Cox regression.
As mentioned in the introduction, tied data poses a problem for Cox regression, but as we have few ties, the Efron approximation should be quite good.

\subsection{Treatment}

We start by fitting a cox regression to the treatment covariate. From the previous log-rank test in Table~\ref{tab:log_rank_indiv}, we found that the treatment and placebo groups are not significantly different, and thus we do not expect the cox regression to be significant either. This is because the score test in cox regression, with categorical covariates, is equivalent to the log-rank test, as long as there are no ties. We do have ties in our data, but not that many.

Table~\ref{tab:cox_treatment} shows some of the results from the cox regression. First note that we display the exponentiated coefficient, as this gives the hazard ratio between the groups, as well as the 95\% confidence interval. We see that the hazard rate is close to one, and 1 is covered in the confidence interval. This indicates that there are no difference between the groups, as expected from our previous analysis.

We can see that the score test in Table~\ref{tab:cox_treatment} is the same as the log-rank test in Table~\ref{tab:log_rank_indiv}, which is as expected.

The columns \verb+zph km p+ and \verb+zph log p+ gives the $p$-value from the \verb+cox.zph+ function in R. This is a test for the proportional hazard assumption, using Schoenfeld-residuals. \verb+km+ and \verb+log+ denotes the Kaplan-Meier and log transform of the survival times. We see from the table than none of the tests suggest that the proportional hazard assumption is false.

\input{tables/cox_treatment}

\subsection{Age}

Next, we fit a Cox regression to the age covariate, yielding the results in Table~\ref{tab:cox_age}. According to the score test, the age covariate is considered significant.  The regression coefficient is quite small, but this is not surprising as the age is coded in years (in stead of e.g.\ decades). It suggest that the hazard ratio between two individuals with a age difference of one year is $1.04$.
None of the zph tests suggest that the proportional hazard assumption is violated.

\input{tables/cox_age}

In Figure~\ref{fig:cox_age_mart} we have plotted the martingale residuals from the cox regression. There does not seem to be a pattern in the residuals, so the functional form of the age covariate seems appropriate. The dense line in the figure shows a GAM fitted to a smoothing spline of the covariate, while the stippled lines give the upper and lower pointwise twice-standard-error curves.

\begin{figure}[h!tb]
    \begin{center}
        \includegraphics[scale=0.6]{./figures/cox_age_mart.pdf}
    \end{center}
    \vspace{-0.2cm}
    \caption{Martingale residuals for cox regression on age.}
    \label{fig:cox_age_mart}
\end{figure}

\subsection{Gender}

Next, we fitted a Cox regression to the gender covariate. As it is categorical, the score test in Table~\ref{tab:cox_gender} gives the same result as the log-rank test in Table~\ref{tab:log_rank_indiv}. In addition we see here that the zph tests suggest that the proportional hazard assumptions seem appropriate.

\input{tables/cox_gender}


\subsection{Bilirubin}

Next, we fit a Cox regression to the bilirubin covariate. Results can be found in Table~\ref{tab:cox_bilirubin}, and the Martingale residuals in Figure~\ref{fig:cox_bil_mart}. The score test give a $p$-value of 0, but the we see a clear pattern in the martingale residuals. We therefore fit a Cox regression with a cubic smoothing-spline to the bilirubin covariate, and got that both the linear and non-linear part was highly significant ($p < 1e-9$, table exempt).

The shape of the martingale residuals might suggest that a log transform of the bilirubin variable could be a better fit. We also plot the spline term against the amount of bilirubin in Figure~\ref{fig:cox_bil_termplot}, where the x-axis is logarithmically scaled. We see that the line is almost linear, supporting our claim that a log transform might be suitable.

Next we fitted a Cox regression to the log transformed bilirubin covariates, yielding the results in Table~\ref{tab:cox_bilirubin_log}.
The martingale residuals were plotted in Figure~\ref{fig:cox_bil_mart_log}, where the residuals are evenly distributed around zero. This again support our log transform.
We see from the zph test that there is nothing suggesting that the proprtional hazard assumption is inappropriate. 

\input{tables/cox_bilirubin}

\begin{figure}[h!tb]
    \begin{center}
        \includegraphics[scale=0.6]{./figures/cox_bil_mart.pdf}
    \end{center}
    \vspace{-0.2cm}
    \caption{Martingale residuals for cox regression on bilirubin.}
    \label{fig:cox_bil_mart}
\end{figure}

\begin{figure}[h!tbp]
    \centering
    \begin{subfigure}[b]{0.48\textwidth}
        \includegraphics[width=\textwidth]{./figures/cox_bil_termplot.pdf}
        \caption{Plot of cox regression term for cubic smoothing-spline. Not that the x-axis is logarithmically scaled.}
        \label{fig:cox_bil_termplot}
    \end{subfigure}%
    \quad
    \begin{subfigure}[b]{0.48\textwidth}
        \includegraphics[width=\textwidth]{./figures/cox_bil_mart_log.pdf}
        \caption{Martingale residuals of cox regression on log bilirubin, and GAM fitted to the residuals.}
        \label{fig:cox_bil_mart_log}
    \end{subfigure}
    %(or a blank line to force the subfigure onto a new line)
    \vspace{1\baselineskip}
    \caption{}
    \label{fig:cox_bil_term_and_mart}
\end{figure}

\input{tables/cox_bilirubin_log}

\subsection{Albumin}

Finally, we fitted a Cox regression to the albumin covariate. From Table~\ref{tab:cox_albumin} we see that the score test consider the albumin covariate as significant.
According to the coefficient, we again see that a lower concentration of albumin is related to higher death rate.
From the martingale residuals in Figure~\ref{fig:cox_alb_mart}, the functional form of the covariate seems appropriate. 
However, the log transform of the proportional hazard test (\verb+zph log p+) is quite low, suggesting that there might be some time dependence in the hazards.
There are different ways of handling this time dependence, but non of them were further pursued here.

\input{tables/cox_albumin}

\begin{figure}[h!tb]
    \begin{center}
        \includegraphics[scale=0.6]{./figures/cox_alb_mart.pdf}
    \end{center}
    \vspace{-0.2cm}
    \caption{Martingale residuals for cox regression on albumin.}
    \label{fig:cox_alb_mart}
\end{figure}



\section{c) Multivariate regression}
\info[inline]{Mention that we tested everything without log-transform, and it was worse}

The last part of this assignment was to fit a multivariate Cox regression to the covariates, and find a model that describes the survival.

As a first step, we found the correlations between the covariates. From Figure~\ref{fig:cor_heat} we see that the covariates are not particularly correlated. The highest (in absolute value) is between bilirubin and albumin on approximately $-0.4$. And from a quick look at all the covariates plotted against each other, there does not seem to be any non-linear effect that cause the low correlations (figures not included).

In this section we only work with the log transformed bilirubin covariate, as we found that is was an appropriate transform in the previous section.
%
\begin{figure}[h!tb]
    \begin{center}
        \includegraphics[scale=0.6]{./figures/cor_heat.pdf}
    \end{center}
    \vspace{-0.8cm}
    \caption{Correlations.}
    \label{fig:cor_heat}
\end{figure}

To find an appropriate set of covariates for the Cox model, we started with all covariates and removed the ones that were not significant one by one. Thus we ended up with a model containing age, bilirubin, and albumin.
After that, we added first order interaction to the models. This was first done for the model with age, bilirubin, and albumin, where we only tested one interaction pair at a time. We then repeated this interaction experiment with gender and treatment (one at a time). We did not use an ANOVA table to check the difference between the models, as all models except one had non-significant (on $5\%$ level) interactions.
The only model with significant interaction was the gender-bilirubin interaction (at $p=0.023$.) However, this made the bilirubin covariate not significant. Investigating Figure~\ref{fig:km_kjonn_bil} we see that there are a lot more women than man in the dataset, which probably cause the gender-bilirubin interaction to just replace the bilirubin covariate. As a result we do not use this model further.
\todo[inline]{Show code for this in appendix}

Our final model therefore contains age, bilirubin, and albumin, without any interactions. We therefore conclude that there is no evidence that the treatment has an effect on the patients.
\todo[inline]{Mention again that age could just be age\ldots}

In Table~\ref{tab:cox_final} and~\ref{tab:cox_final_ci} we show the resuting coefficients with confidence intervals and $p$-values. We see that the coefficient for age has not changes from the univariate case, while the coefficients for bilirubin and albumin has changed more. This probably because there is a stronger dependence between bilirubin and albumin.

\todo[inline]{Interpret coefficients}

\input{./tables/cox_final}

\input{./tables/cox_final_ci}

The score test of the final model has a $p$-value $< 1e-16$.

In Table~\ref{tab:cox_time_dep} and~\ref{tab:cox_time_dep_log} we have tested the proportional hazard assumption of the model, using Schoenfeld-residuals. We see that non of the test indicate that there is a time dependency in the hazards. However, we find lowest $p$-values for the albumin covariate, so it could be interesting to further investigate this. This was, however, not pursued in this project.

\input{./tables/cox_time_dep}

\input{./tables/cox_time_dep_log}

In Figure~\ref{fig:cox_full_linearity}, we have plotted the martingale residuals against the individual covariates. The residual seem to be evenly distributed around zero, suggesting that the functional form of our covariates are appropriate.

\begin{figure}[h!tb]
    \begin{center}
        \includegraphics[scale=0.8]{./figures/cox_full_linearity.pdf}
    \end{center}
    \vspace{-0.8cm}
    \caption{Martingale residuals for the full model. The lines are a GAM fitted to a smoothing spline of the relevant covariate. The bounds are the upper and lower pointwise twice-standard-error curves.}
    \label{fig:cox_full_linearity}
\end{figure}











%%%%%%%%%%%%%%%%%%%%%%%%%%%%%%%%%%%%%%%%%%%%%%%%%%%%%%%%%%%%%%%%%%%%%%%%%%%%%%%%
\cleardoublepage{}
\phantomsection
\addcontentsline{toc}{chapter}{Bibliography}
%\bibliographystyle{plain}
%\bibliographystyle{apa}
%\bibliographystyle{authordate4}
%\bibliographystyle{ksfh_nat}
\bibliographystyle{plainnat}
\bibliography{bibliography}

%
%%%%%%%%%%%%%%%%%%%%%%%%%%%%%%%%%%%%%%%%%%%%%%%%%%%%%%%%%%%%%%%%%%%%%%%%%%%%%%%%
%\clearpage
%\begin{appendices}
  %% Use \section
  %% Can use \phantomsection in text to get back
%\end{appendices}
%%%%%%%%%%%%%%%%%%%%%%%%%%%%%%%%%%%%%%%%%%%%%%%%%%%%%%%%%%%%%%%%%%%%%%%%%%%%%%%%
%\begin{thebibliography}{99} % At most 99 references.
%% Use "bibit" to generate bibitem
%\end{thebibliography}
%
\end{document}
