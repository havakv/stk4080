\documentclass[11pt,a4paper]{article}
\linespread{1.25}
\usepackage[utf8]{inputenc} % Norwegian letters
\usepackage{natbib}
\usepackage{amsmath}
\usepackage{float}  % Used for minipage and stuff.
\usepackage{wrapfig} % Wrap text around figure wrapfig [tab]
\usepackage{graphicx}
\usepackage{enumerate} % Use e.g. \begin{enumerate}[a)]
\usepackage[font={small, it}]{caption} % captions on figures and tables
\usepackage{subcaption}
\usepackage[toc,page]{appendix} % Page make Appendices title, toc fix table of content 
%\usepackage{todonotes} % Notes. Use \todo{"text"}. Comment out \listoftodos
\usepackage{xargs}                      % Use more than one optional parameter in a new commands
\usepackage[pdftex,dvipsnames]{xcolor}  % Coloured text etc.
% 
\usepackage[colorinlistoftodos,prependcaption,textsize=footnotesize]{todonotes}
\newcommandx{\unsure}[2][1=]{\todo[linecolor=red,backgroundcolor=red!25,bordercolor=red,#1]{#2}}
\newcommandx{\change}[2][1=]{\todo[linecolor=blue,backgroundcolor=blue!25,bordercolor=blue,#1]{#2}}
\newcommandx{\info}[2][1=]{\todo[linecolor=OliveGreen,backgroundcolor=OliveGreen!25,bordercolor=OliveGreen,#1]{#2}}
\newcommandx{\improvement}[2][1=]{\todo[linecolor=Plum,backgroundcolor=Plum!25,bordercolor=Plum,#1]{#2}}
\newcommandx{\thiswillnotshow}[2][1=]{\todo[disable,#1]{#2}}
\usepackage{microtype} % Improves spacing. Include AFTER fonts
\usepackage{hyperref} % Use \autoref{} and \nameref{}
\hypersetup{backref,
  colorlinks=true,
  breaklinks=true,
  %hidelinks, %uncomment to make links black
  citecolor=blue,
  linkcolor=blue,
  urlcolor=blue
}
\usepackage[all]{hypcap} % Makes hyperref jup to top of pictures and tables
%
%-------------------------------------------------------------------------------
% Page layout
%\usepackage{showframe} % Uncomment if you want the margin frames
\usepackage{fullpage}
\topmargin=-0.25in
%\evensidemargin=-0.3in
%\oddsidemargin=-0.3in
%\textwidth=6.9in
%\textheight=9.5in
\headsep=0.25in
\footskip=0.50in

%-------------------------------------------------------------------------------
% Header and footer
\usepackage{lastpage} % To be able to add last page in footer.
\usepackage{fancyhdr} % Custom headers and footers
\pagestyle{fancy} % Use "fancyplain" for header in all pages
%\renewcommand{\chaptermark}[1]{ \markboth{#1}{} } % Usefull for book?
\renewcommand{\sectionmark}[1]{ \markright{\thesection\ #1}{} } % Remove formating and nr.
%\fancyhead[LE, RO]{\footnotesize\leftmark}
%\fancyhead[RO, LE]{\footnotesize\rightmark}
\lhead[]{\AuthorName}
\rhead[]{\rightmark}
\fancyfoot[L]{} % Empty left footer
\fancyfoot[C]{} % Empty center footer
\fancyfoot[R]{Page\ \thepage\ of\ \protect\pageref*{LastPage}} % Page numbering for right footer
\renewcommand{\headrulewidth}{1pt} % header underlines
\renewcommand{\footrulewidth}{1pt} % footer underlines
\setlength{\headheight}{13.6pt} % Customize the height of the header

%-------------------------------------------------------------------------------
% Suppose to make it easier for LaTeX to place figures and tables where I want.
\setlength{\abovecaptionskip}{0pt plus 1pt minus 2pt} % Makes caption come closer to figure.
%\setcounter{totalnumber}{5}
%\renewcommand{\textfraction}{0.05}
%\renewcommand{\topfraction}{0.95}
%\renewcommand{\bottomfraction}{0.95}
%\renewcommand{\floatpagefraction}{0.35}
%
% Math short cuts for expectation, variance and covariance
\newcommand{\E}{\mathrm{E}}
\newcommand{\Var}{\mathrm{Var}}
\newcommand{\Cov}{\mathrm{Cov}}
\newcommand{\mk}[1]{\colorbox{yellow}{#1}}
% Commands for argmin and argmax
\DeclareMathOperator*{\argmin}{arg\,min}
\DeclareMathOperator*{\argmax}{arg\,max}
%%%%%%%%%%%%%%%%%%%%%%%%%%%%%%%%%%%%%%%%%%%%%%%%%%%%%%%%%%%%%%%%%%%%%%%%%%%%%%%%
%-----------------------------------------------------------------------------
%	TITLE SECTION
%-----------------------------------------------------------------------------
\newcommand{\AuthorName}{Håvard Kvamme} % Your name

\newcommand{\horrule}[1]{\rule{\linewidth}{#1}} % Create horizontal rule command with 1 argument of height

\title{\
\normalfont \normalsize 
\textsc{STK9101SP} \\ [25pt] % Your university, school and/or department name(s)
\horrule{0.5pt} \\[0.4cm] % Thin top horizontal rule
\huge Project on Survival Analysis \\ % The assignment title
\horrule{2pt} \\[0.5cm] % Thick bottom horizontal rule
}

\author{\AuthorName} % Your name

\date{\normalsize\today} % Today's date or a custom date
\begin{document}
\maketitle
%%%%%%%%%%%%%%%%%%%%%%%%%%%%%%%%%%%%%%%%%%%%%%%%%%%%%%%%%%%%%%%%%%%%%%%%%%%%%%%%
%\listoftodos{}
%
\section{Introduction}
\todo[inline]{Introduce problem and covariates}

The dataset has tied events due to the rounding of events to nearest date. Out of the 307 observations, 296 occurs alone in a single date. However, this does not affect the estimation of the Kaplan-Meier.


\section{a) Simple univariate analysis}
\unsure[inline]{Does univariate mean only one group??? \\I would prefer to use many several groups.}
\unsure[inline]{Should I try to interpret the K-M plots through quantiles and stuff?? I’m pretty sure I’ve seen that somewhere. Show median survival times for different groups?}


\begin{figure}[h!tbp]
    \centering
    \begin{subfigure}[b]{0.48\textwidth}
        \includegraphics[width=\textwidth]{./figures/km_beh.pdf}
        %\caption{Alb.}
        %\label{fig:cartAreas1}
    \end{subfigure}%
    \quad
    \begin{subfigure}[b]{0.48\textwidth}
        \includegraphics[width=\textwidth]{./figures/km_ald.pdf}
        %\caption{not alb.}
        %\label{fig:cartTree1}
    \end{subfigure}
    %(or a blank line to force the subfigure onto a new line)
    \vspace{1\baselineskip}
    \caption{Survival curves for treatment and age. We have included histograms for the two covariates under the survival curves.}
    \label{fig:alb}
\end{figure}

\begin{figure}[h!tbp]
    \centering
    \begin{subfigure}[b]{0.48\textwidth}
        \includegraphics[width=\textwidth]{./figures/km_kjonn.pdf}
        %\caption{Alb.}
        %\label{fig:cartAreas1}
    \end{subfigure}%
    \quad
    \begin{subfigure}[b]{0.48\textwidth}
        \includegraphics[width=\textwidth]{./figures/km_bil.pdf}
        %\caption{not alb.}
        %\label{fig:cartTree1}
    \end{subfigure}
    %(or a blank line to force the subfigure onto a new line)
    \vspace{1\baselineskip}
    \caption{Survival curves for gender and amount of bilirubin in the blood of the test subjects. We have included histograms for the two covariates under the survival curves.}
    \label{fig:alb}
\end{figure}

\begin{figure}[h!tb]
    \begin{center}
        \includegraphics[scale=0.8]{./figures/km_alb.pdf}
    \end{center}
    \vspace{-0.8cm}
    \caption{Survival curve for amount of albumin in the blood of the test subjects. The histogram displays the distribution of albumin among the subjects.}
    \label{fig:alb}
\end{figure}

\begin{table}[h!tbp]
    \centering
    \caption{Log-rank tests on the individual covariates.}
    \label{tab:log_rank_indiv}
    \begin{tabular}{rrrr}
        \hline
        Covariate & deg. & Chisq. & $p$ \\ 
        \hline
        Treatment &  1 &  $0   $ & $0.853     $ \\
        Age       &  3 &  $21  $ & $1.07e-4   $ \\ 
        Gender    &  1 &  $3.5 $ & $0.0613    $ \\ 
        Bilirubin &  4 &  $182 $ & $0         $ \\ 
        Albumin   &  3 &  $68.7$ & $7.99e-15$ \\ 
        \hline
    \end{tabular}
\end{table}

\section{b) Univariate regression}
\info[inline]{For categorical covariates, cox scores and log-rank are equivalent, if not ties. What with ties??}
\unsure[inline]{How does cox handle tied data?}
\info[inline]{Cox can fail if $x$ is wrong and we should use log or sqrt instead. Or if the hazards are not proportional.}

\subsection{Treatment}

Here we fitted an cox regression to the treatment covariate. We see from Table~\ref{tab:log_rank_indiv} that the treatment and placebo group are not significantly different, and thus we do not expect the cox regression to be significant either. This is because the score test in cox regression, with categorical covariates, is equivalent to the log-rank test, as long as there are no ties. We do have ties in our data, but not that many.

Table~\ref{tab:cox_treatment} shows some of the results from the cox regression. First note that we display the exponentiated coefficient, as this gives the hazard ratio between the groups, as well as the 90\% confidence interval. We see that the hazard rate is close to one, and 1 is covered in the confidence interval. This indicates that there are no difference between the groups, as expected from our previous analysis.

We can see that the score test in Table~\ref{tab:cox_treatment} is the same as the log-rank test in Table~\ref{tab:log_rank_indiv}, which is as expected.

Also note that none of the tests for the proportional hazard assumption fails (zph).
So the cox model seems to be a reasonable model for the treatment covariate.

\input{tables/cox_treatment}


\subsection{Age}

\input{tables/cox_age}

In Figure~\ref{fig:cox_age_mart} we have plotted the martingale residuals from the cox regression of age. There does not seem to be a pattern in the residuals, so the functional form of the covariate seems appropriate.

\begin{figure}[h!tb]
    \begin{center}
        \includegraphics[scale=0.6]{./figures/cox_age_mart.pdf}
    \end{center}
    \vspace{-0.2cm}
    \caption{Martingale residuals for cox regression on age.}
    \label{fig:cox_age_mart}
\end{figure}

\subsection{Gender}
\input{tables/cox_gender}


\subsection{Bilirubin}

We fit a cox regression to the bilirubin covariate, and the results can be found in Table~\ref{tab:cox_bilirubin}, and the Martingale residuals in Figure~\ref{fig:cox_bil_mart}. The score test give a $p$-value of 0, but the we see a clear pattern in the martingale residuals. We therefore fit a cox regression with a cubic smoothing-spline to the bilirubin covariate, and got that both the linear and non-linear part was highly significant ($p < 1e-9$, table exempt).

The shape of the martingale residuals might suggest that a log-transform of the bilirubin variable could be a better fit. We also plot the spline term against the amount of bilirubin in Figure~\ref{fig:cox_bil_termplot}, where the x-axis is logarithmically scaled. We see that the line is almost linear, supporting our claim that a log-transform might be suitable.

Next we fitted a cox regression to the log transformed bilirubin covariates, yielding the results in Table~\ref{tab:cox_bilirubin_log}.

The martingale residuals were plotted in Figure~\ref{fig:cox_bil_mart_log}, where the residuals are evenly distributed around zero. This again support our log-transform.

\input{tables/cox_bilirubin}
\unsure[inline]{Something on how log transform is reasonable.}

\begin{figure}[h!tb]
    \begin{center}
        \includegraphics[scale=0.6]{./figures/cox_bil_mart.pdf}
    \end{center}
    \vspace{-0.2cm}
    \caption{Martingale residuals for cox regression on bilirubin.}
    \label{fig:cox_bil_mart}
\end{figure}

\begin{figure}[h!tbp]
    \centering
    \begin{subfigure}[b]{0.48\textwidth}
        \includegraphics[width=\textwidth]{./figures/cox_bil_termplot.pdf}
        \caption{Plot of cox regression term for cubic smoothing-spline. Not that the x-axis is logarithmically scaled.}
        \label{fig:cox_bil_termplot}
    \end{subfigure}%
    \quad
    \begin{subfigure}[b]{0.48\textwidth}
        \includegraphics[width=\textwidth]{./figures/cox_bil_mart_log.pdf}
        \caption{Martingale residuals of cox regression on log bilirubin, and GAM fitted to the residuals.}
        \label{fig:cox_bil_mart_log}
    \end{subfigure}
    %(or a blank line to force the subfigure onto a new line)
    \vspace{1\baselineskip}
    \caption{}
    \label{fig:cox_bil_term_and_mart}
\end{figure}

\input{tables/cox_bilirubin_log}

\subsection{Albumin}
\input{tables/cox_albumin}

\begin{figure}[h!tb]
    \begin{center}
        \includegraphics[scale=0.6]{./figures/cox_alb_mart.pdf}
    \end{center}
    \vspace{-0.2cm}
    \caption{Martingale residuals for cox regression on albumin.}
    \label{fig:cox_alb_mart}
\end{figure}


\subsection{Testing model assumptions}

\begin{itemize}
    \item Can plot martingale residuals vs time (observation number) and color by group. To see if the groups are evenly spread. This check the PH assumption.
    \item Investigate martingale residuals. Figure this out!!!!
    \item Scaled Schoenfeld residuals (lec 9 p 35) to check proportionality over time.
    \item Schoenfeld residuals for prop. assumption (lec 9 p 38).
    \item Check proportionality by lec 9 p 29: Plot log nelson aalen for different groups and check if they are parallel.
    \item Practical exercise 8. Fit spline to check if log-linear effect (which is a model assumption?). How to interpret results?
    \item \url{http://www.mwsug.org/proceedings/2006/stats/MWSUG-2006-SD08.pdf}
\end{itemize}



\section{c) Multivariate regression}
\begin{itemize}
    \item See Practical exercise 7 for likelihood ratio test for significance of interactions.
\end{itemize}


We see from Figure~\ref{fig:cor_heat} that the covariates are not particularly correlated. The highest (in absolute value) is between bilirubin and albumin on approximately $-0.4$. And from a quick look at all the covariates plotted against each other, there does not seem to be any non-linear effect that cause the low correlations (figures not included).
%
\begin{figure}[h!tb]
    \begin{center}
        \includegraphics[scale=0.6]{./figures/cor_heat.pdf}
    \end{center}
    \vspace{-0.8cm}
    \caption{Correlations.}
    \label{fig:cor_heat}
\end{figure}

We started with all covariates and removed the ones that were not significant (just treatment).

To test for inclusion of 












%%%%%%%%%%%%%%%%%%%%%%%%%%%%%%%%%%%%%%%%%%%%%%%%%%%%%%%%%%%%%%%%%%%%%%%%%%%%%%%%
\cleardoublepage{}
\phantomsection
\addcontentsline{toc}{chapter}{Bibliography}
%\bibliographystyle{plain}
%\bibliographystyle{apa}
%\bibliographystyle{authordate4}
%\bibliographystyle{ksfh_nat}
\bibliographystyle{plainnat}
\bibliography{bibliography}

%
%%%%%%%%%%%%%%%%%%%%%%%%%%%%%%%%%%%%%%%%%%%%%%%%%%%%%%%%%%%%%%%%%%%%%%%%%%%%%%%%
%\clearpage
%\begin{appendices}
  %% Use \section
  %% Can use \phantomsection in text to get back
%\end{appendices}
%%%%%%%%%%%%%%%%%%%%%%%%%%%%%%%%%%%%%%%%%%%%%%%%%%%%%%%%%%%%%%%%%%%%%%%%%%%%%%%%
%\begin{thebibliography}{99} % At most 99 references.
%% Use "bibit" to generate bibitem
%\end{thebibliography}
%
\end{document}
